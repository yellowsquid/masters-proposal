% Created 2021-11-18 Thu 12:35
% Intended LaTeX compiler: pdflatex
\documentclass[11pt]{article}
\usepackage[utf8]{inputenc}
\usepackage[T1]{fontenc}
\usepackage{graphicx}
\usepackage{longtable}
\usepackage{wrapfig}
\usepackage{rotating}
\usepackage[normalem]{ulem}
\usepackage{amsmath}
\usepackage{amssymb}
\usepackage{capt-of}
\usepackage{hyperref}
\usepackage[hyperref=true,url=true,backend=biber,natbib=true]{biblatex}
\usepackage{a4wide}
\usepackage{parskip}
\usepackage{times}
\addbibresource{proposal.bib}
\setcounter{secnumdepth}{1}
\date{}
\title{}
\hypersetup{
 hidelinks=true,
 pdfauthor={},
 pdftitle={},
 pdfkeywords={},
 pdfsubject={},
 pdfcreator={Emacs 28.0.60 (Org mode 9.5)}, 
 pdflang={British}}
\begin{document}

\centerline{\Large Semantics of an embedded vector architecture for formal verification of software}
\vspace{2em}
\centerline{\Large \emph{A Part III project proposal}}
\vspace{2em}
\centerline{\large G. M. Brown (\emph{gmb60}), Queens' College}
\vspace{1em}
\centerline{\large Project Supervisor: Dr J. D. Yallop}
\vspace{1em}

\begin{abstract}
All good implementations of cryptographic algorithms must be correct,
side-channel free and fast. Most cryptographic libraries focus on maximising 
speed by writing hand-tuned assembly. This can introduce subtle bugs that
invalidate correctness or introduce side-channels. Whilst tools exist to help
formally verify these algorithms, none are designed to target the recent
M-profile Vector Extension for the Armv8.1-M architecture. My project seeks to
define semantics for these vector instructions, designed to be used for formal
verification of software. I will use these semantics to formally verify the
correctness of hand-written assembly for cryptographic applications.
\end{abstract}

\section{Introduction, approach and outcomes}
\label{sec:org73235ed}

In almost all cases, the best implementation of an algorithm will be both
correct and fast. If an implementation is not correct, then it implements a
different algorithm. If an implementation is slow, then it wastes resources
that could be used to perform additional work. Ensuring an implementation
satisfies both properties is usually difficult.

This is especially true when it comes to writing assembly code, to maximise
performance. Truly optimal performance comes when details of the processor
microarchitecture are used to reorder instructions to minimise the amount time
the processor is stalled. After shuffling the instructions, it can be
incredibly difficult to recognise the original algorithm and the intent of
each instruction. This leads to subtle bugs which can invalidate the
correctness and other safety properties of the implementation.

To eradicate the correctness bugs with certainty, formal verification is
necessary. Formal verification starts with a description of the semantics of
machine code --- a model of the action of instructions on machine state. This
semantic framework is incorporated into a larger logical system, such as a
higher-order-logic theorem prover like Isabelle or a dependently-typed
programming language like Agda. These logic systems can then be used to
formally prove that given assembly code satisfies the desired correctness
properties, along with any other safety property. For instance, formal
verification is often used in cryptography to prove the absence of
side-channels.

Formal verification tools already exist for some architectures. For example,
Jasmin is a tool for x86 that accepts assembly language augmented with logical
assertions and high-level constructs like variables and loops. This is
converted into pure assembly language output, as well as a proof that all the
assertions in the input hold.

An often-overlooked architecture in the field of formal verification is
Armv8.1-M. This instruction set is designed for use by microcontrollers, which
operate in a resource-constrained environment. In particular the recent
M-profile Vector Extension (MVE), which provides SIMD instructions for
Armv8.1-M cores, has no known semantics suitable for the formal verification
of software.

My project is to resolve this issue by using Agda to develop a semantics for
MVE instructions suitable for formal verification. This entails the design of
operational and axiomatic semantics, and proof of an equivalence between these
two approaches. Operational semantics describe the explicit effect of
instructions on the state of the machine, whereas axiomatic semantics describe
how execution effects properties of the state.

I will also use the two semantic systems to formally verify the correctness of
several algorithms used in cryptography, with implementations produced by
Becker et al.\ \cite{cryptoeprint_2021_998}. These include Barrett reduction,
Montgomery multiplication and the number-theoretic transform.

\section{Workplan}
\label{sec:org96415de}

\begin{center}
\begin{tabular}{rrp{0.7\linewidth}}
Start & End & Work\\
\hline
2021-12-06 & 2021-12-19 & Familiarise myself with MVE instructions. Begin work on defining the operational semantics, focusing on representing machine state and auxiliary definitions.\\
2021-12-20 & 2022-01-02 & Break for Christmas and New Year's.\\
2022-01-03 & 2022-01-16 & Finish the definition of the operational semantics. Prove some useful lemmas regarding manipulating the structure of blocks of assembly.\\
2022-01-17 & 2022-01-30 & Define the axiomatic semantics in the form of Hoare triples.\\
2022-01-31 & 2022-02-13 & Prove and equivalence between the operational and axiomatic semantics.\\
2022-02-14 & 2022-02-27 & Progress review. Begin formal verification of Barrett reduction.\\
2022-03-14 & 2022-03-27 & Formally verify an implementation of Montgomery multiplication.\\
2022-03-14 & 2022-03-27 & Begin work to formally verify one iteration of the number-theoretic transform, which is a form of discrete Fourier transform in the domain of integers modulo a prime.\\
2022-03-28 & 2022-04-10 & Finish verification of one iteration of the number-theoretic transform.\\
2022-04-11 & 2022-04-24 & Write-up.\\
2022-04-25 & 2022-05-08 & Write-up.\\
2022-05-09 & 2022-05-22 & Contingency weeks.\\
2022-05-23 & 2022-05-27 & Submission. Begin work on presentation.\\
\end{tabular}

\end{center}

\newpage
\appendix
\printbibliography{}
\end{document}